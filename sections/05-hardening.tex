\section{Kernal Hardening}
\label{Kernal Hardening}
\label{AppArmor}
\addcontentsline{toc}{subsection}{AppArmor}
\begin{baseBoxThree}{\sL{AppArmor}}{dark}
    \bigskip
    \sTE{Description}: AppArmor (Application Armor) is a Linux security module that aims to protect the operating system and its applications by enforcing access control policies.
    Unlike SELinux, which uses labels, AppArmor uses a simpler, path-based approach to control how applications and processes can access files, network resources, and other system components.
    Essentially, AppArmor adds an additional layer of security by specifying what a program can and cannot do, thus enhancing the traditional Linux permissions system.
    \bigskip
    \begin{baseBoxThree}{}{dark}
        \begin{posnexItemize}
            \item[\sA] \sE{Enhanced Security}: AppArmor enforces strict access control policies, reducing the risk of unauthorized access and potential security breaches.
            \item[\sA] \sE{Profile-Based}: AppArmor uses profiles to define what actions different users and applications are allowed to perform. These profiles can be customized to meet specific security requirements.
            \item[\sA] \sE{Path-Based Controls}: Files, processes, and other objects are assigned security contexts based on their paths, which are used to enforce policies.
            \item[\sA] \sE{Enforcement Modes}: AppArmor operates in different modes, including enforcing (actively enforcing policies) and complain (logging policy violations without enforcing them).
        \end{posnexItemize}
        \smallskip
        \begin{baseBoxThree}{}{dark}
            \smallskip
            AppArmor is particularly valuable in environments where security is a top priority, such as government systems, financial institutions, and other organizations handling sensitive data.
            \smallskip
        \end{baseBoxThree}
        \smallskip
    \end{baseBoxThree}
    \smallskip
    \begin{baseBoxThree}{Setting Up AppArmor on Alpine Linux}{dark}
        \smallskip
        Since Alpine Linux does not come with AppArmor pre-configured, we will guide you through the process of setting up AppArmor from scratch, similar to the approach used in Arch Linux.
        \smallskip
    \end{baseBoxThree}
    \smallskip
\end{baseBoxThree}

\begin{baseBoxThree}{\fU{AppArmor}}{dark}
    \smallskip
    \begin{baseBoxThree}{What We'll Be Doing}{dark}
        \smallskip
        \begin{posnexItemize}
            \item[\sA] \sE{Installing AppArmor Packages}: We'll start by installing the necessary AppArmor tools and utilities.
            \item[\sA] \sE{Configuring Boot Parameters}: Modifying the bootloader configuration to enable AppArmor during system startup.
            \item[\sA] \sE{AppArmor Configurations}: Creating and customizing AppArmor configuration files.
            \item[\sA] \sE{Applying AppArmor Profiles}: Applying AppArmor profiles to applications, which is crucial for enforcing policies.
            \item[\sA] \sE{Loading AppArmor Policies}: Installing and loading default AppArmor policies to ensure proper security enforcement.
        \end{posnexItemize}
        \smallskip
        \begin{baseBoxThree}{Why This Matters}{dark}
            \smallskip
            By implementing SELinux on Alpine Linux, we're significantly enhancing the overall security posture of our system. SELinux will help us control access to files, processes, and network resources more effectively, reducing the risk of security breaches and protecting sensitive data.
            Stay tuned as we dive into each of these steps in detail and transform our Alpine Linux setup into a fortified and secure environment!
            \smallskip
        \end{baseBoxThree}
        \smallskip
    \end{baseBoxThree}
    \smallskip
    \begin{baseBoxThree}{Note}{info}
        \smallskip
        This guide assumes that you are using \sTE{Alpine Linux} and the \sTE{extlinux bootloader}. If you are using Alpine Linux, please check to ensure that you are indeed using extlinux.
        \smallskip
    \end{baseBoxThree}
        \smallskip
\end{baseBoxThree}

\begin{baseBoxThree}{\fU{AppArmor}}{dark}
    \smallskip
    \begin{baseBoxThree}{Setup}{dark}
        \begin{posnex}
# Update the package list
apk -U update

# Install AppArmor and AppArmor utilities
apk add apparmor apparmor-utils apparmor-profiles

# Edit extlinux.conf to append 'apparmor' to the lsm parameter
# Creates a backup of the original file with a timestamp
sed -i_$(date +"%s").bak '
        /APPEND /{
                # Remove any existing lsm parameters
                s/ lsm=.*//;
                # Append the current security modules and add 'apparmor'
                s/$/ lsm='$(awk -F ',' '{
                        for (i = 1; i < NF; i++) {
                                if ($i !~ /^(landlock|yama|apparmor)$/)
                                        printf("%s,", $i)
                        }
                }' /sys/kernel/security/lsm)'landlock,yama,apparmor/
        }
' /boot/extlinux.conf

lockdown,capability,landlock,yama
# Update the extlinux bootloader configuration
update-extlinux 

# Start the AppArmor service
rc-service apparmor start

# Enable AppArmor to start on boot
rc-update add apparmor boot

# Check if AppArmor is enabled
aa-enabled
        \end{posnex}            
    \end{baseBoxThree}
    \smallskip
\end{baseBoxThree}

