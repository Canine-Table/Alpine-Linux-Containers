\phantomsection{}
\label{Stack Canaries}
\addcontentsline{toc}{subsection}{Stack Canaries}
\begin{baseBoxThree}{\sL{Stack Canaries}}{dark}
    \bigskip
    \sTE{Description}: Stack canaries are a security mechanism used to detect and prevent stack buffer overflow attacks, which occur when a program writes more data to a buffer on the stack than it can hold.
    \bigskip
    \begin{baseBoxThree}{Breakdown}{dark}
        \smallskip
        \begin{posnexItemize}
            \item[\sA] \sE{Canary Value}: A small, known value (the "canary") is placed between the stack's control data (such as the return address) and the local variables.
            \item[\sA] \sE{Checking the Canary}: Before a function returns, the canary value is checked to see if it has been altered.
            \item[\sA] \sE{Detection of Overflow}: If a buffer overflow occurs and overwrites the canary value, the program detects the change and typically responds by terminating or taking other protective actions.
        \end{posnexItemize}
        \smallskip
        The goal of stack canaries is to prevent the attacker from overwriting important control data, such as return addresses, without being detected.
        \smallskip
    \end{baseBoxThree}
    \smallskip
    \phantomsection{}
    \label{Types of Canaries}
    \begin{baseBoxThree}{Types}{dark}
        \smallskip
        \begin{posnexItemize}
            \item[\sA] \fD{Terminator Canaries}: These canaries contain null bytes, carriage returns, and newline characters, making them difficult to overwrite with standard string functions.
            \item[\sA] \sE{Random Canaries}: These canaries are randomly generated at program startup, making them unpredictable and harder for an attacker to guess.
            \item[\sA] \sE{Guard Pages}: These canaries use memory pages marked as non-readable or non-writable to detect stack overflows.
        \end{posnexItemize}
        \bigskip
        Stack canaries are an important part of modern software security practices, helping to protect against certain types of buffer overflow attacks.
        \smallskip
    \end{baseBoxThree}
    \bigskip
    The script below demonstrates how to create and run two C programs—one with stack canaries enabled and one without.
    The programs are designed to illustrate how stack canaries can detect and prevent buffer overflow attacks.
    \smallskip
\end{baseBoxThree}

\begin{baseBoxThree}{\fU{Stack Canaries}}{dark}
    \smallskip
    \phantomsection{}
    \label{Terminator Canaries}
    \begin{baseBoxThree}{\fUD{Types of Canaries}{Example: Terminator Canary}}{dark}
        \begin{posnex}
apk update && apk upgrade
apk add gcc build-base
cd /tmp

{
cat << 'EOF'
#include <stdio.h>
#include <string.h>

void vulnerable_function(char*);

int main(int argc, char **argv) {
        if (argc != 2) {
                fprintf(stderr, "Usage: %s <input string>\n", argv[0]);
                return 1;
        }

        vulnerable_function(argv[1]);
        printf("Function executed successfully.\n");
        return 0;
}

void vulnerable_function(char *str) {
        char buffer[16];
        strcpy(buffer, str); // This is intentionally vulnerable
} 
EOF
} > vulnerable_example.c

# Compile the file without stack canaries
gcc -fno-stack-protector -o example_without_canary vulnerable_example.c

# Compile the file with stack canaries enabled (default)
gcc -o example_with_canary vulnerable_example.c

# Run the program with stack canaries
# (this should detect the overflow) 
./example_with_canary "this_is_a_very_long_input_string"

# Run the program without stack canaries
# (this might crash or behave unexpectedly)
./example_without_canary "this_is_a_very_long_input_string"
        \end{posnex}
    \end{baseBoxThree}
\end{baseBoxThree}

\begin{baseBoxThree}{\fU{Stack Canaries}}{dark}
    \smallskip
    \begin{baseBoxThree}{Note}{info}
        \smallskip
            Since this is Alpine, you should expect the program without stack canaries to result in a core dump when a buffer overflow occurs.
            This demonstrates the robustness of Alpine's handling of such vulnerabilities.
        \smallskip
    \end{baseBoxThree}
    \smallskip
\end{baseBoxThree}
