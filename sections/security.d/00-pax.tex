\phantomsection{}
\label{PaX}
\addcontentsline{toc}{subsection}{PaX}
\begin{baseBoxThree}{\sL{PaX}}{dark}
    \bigskip
    \sTE{Description}: PaX provides a set of security enhancements originally derived from the PaX patches for the Linux kernel.
    These include various protections against memory corruption vulnerabilities.
    \bigskip
    \phantomsection{}
    \label{Open Flags}
    \begin{baseBoxThree}{Pax Flags}{info}
        \smallskip
        \begin{posnexItemize}
            \item[\sA] \fD{PAGEEXEC}: Prevents execution of code on writable pages (stack and heap). 
            This is one of the key protections against code injection attacks, ensuring that even if an attacker manages to inject malicious code, it cannot be executed.
            \item[\sA] \fD{EMUTRAMP}: Emulates trampolines to allow certain legitimate uses of writable and executable pages.
            This flag makes exceptions for specific, safe cases where executing writable pages is necessary, like for some just-in-time (JIT) compilers.
            \item[\sA] \fD{MPROTECT}: Forces the use of mprotect to set memory protections, ensuring that memory protections are enforced.
            By restricting permissions changes to memory pages, it prevents common exploits that rely on making a page writable and executable.
            \item[\sA] \fD{RANDMMAP}: Implements Address Space Layout Randomization (ASLR) to randomize the memory layout of processes, making it harder for attackers to predict addresses.
            This increases the complexity of attacks by making it difficult for attackers to predict the location of specific functions or variables in memory.
            \item[\sA] \fD{SEGMEXEC}: Allows the execution of code in the data segment, providing an additional layer of protection against buffer overflow attacks.
            This protection creates separate memory segments for executable and non-executable data, reducing the risk of executing malicious code.
        \end{posnexItemize}
        \smallskip
    \end{baseBoxThree}
    \smallskip
\end{baseBoxThree}

\begin{baseBoxThree}{\fU{PaX}}{dark}
    \smallskip
    \begin{baseBoxThree}{PaX Setup}{dark}
        \begin{posnex}
# Ensure pax-utils and paxctl are installed in the container
apk update && apk upgrade
apk add pax-utils paxctl

# Convert the PT_GNU_STACK program header to PT_PAX_FLAGS for busybox
paxctl -c /bin/busybox

# Check the current PaX flags for the busybox executable
paxctl -v /bin/busybox
        \end{posnex}
    \end{baseBoxThree}
    \smallskip
    \phantomsection{}
    \label{PAGEEXEC}
    \begin{baseBoxThree}{\fUD{Open Flags}{PAGEEXEC}}{dark}
        \begin{posnex}
# Enable PAGEEXEC for an executable
# This ensures that code cannot be executed from writable pages
paxctl -P busybox

# Disable PAGEEXEC for an executable
# This allows execution of code from writable pages (not recommended for security)
paxctl -p busybox
        \end{posnex}
    \end{baseBoxThree}
    \smallskip
    \phantomsection{}
    \label{EMUTRAMP}
    \begin{baseBoxThree}{\fUD{Open Flags}{EMUTRAMP}}{dark}
        \begin{posnex}
# Enable EMUTRAMP for an executable
# This allows certain legitimate uses of writable and executable pages
paxctl -E busybox

# Disable EMUTRAMP for an executable
# This disables the emulation of trampolines
paxctl -e busybox
        \end{posnex}
    \end{baseBoxThree}
    \smallskip
\end{baseBoxThree}

\begin{baseBoxThree}{\fU{PaX}}{dark}
    \phantomsection{}
    \label{MPROTECT}
    \begin{baseBoxThree}{\fUD{Open Flags}{MPROTECT}}{dark}
        \begin{posnex}
# Enable MPROTECT for an executable
# This forces the use of mprotect to set memory protections
paxctl -M busybox

# Disable MPROTECT for an executable
# This disables the enforcement of memory protections
paxctl -m busybox
        \end{posnex}
    \end{baseBoxThree}
    \smallskip
    \phantomsection{}
    \label{RANDMMAP}
    \begin{baseBoxThree}{\fUD{Open Flags}{RANDMMAP}}{dark}
        \begin{posnex}
# Enable RANDMMAP for an executable
# This implements ASLR to randomize the memory layout of processes
paxctl -R busybox

# Disable RANDMMAP for an executable
# This disables the randomization of memory addresses
paxctl -r busybox
        \end{posnex}
    \end{baseBoxThree}
    \smallskip
    \phantomsection{}
    \label{RANDEXEC}
    \begin{baseBoxThree}{\fUD{Open Flags}{RANDEXEC}}{dark}
        \begin{posnex}
# Enable RANDEXEC for an executable
# This randomizes the locations of executable segments
paxctl -X busybox

# Disable RANDEXEC for an executable
# This disables the randomization of executable segments
paxctl -x busybox
        \end{posnex}
    \end{baseBoxThree}
    \smallskip
    \phantomsection{}
    \label{SEGMEXEC}
    \begin{baseBoxThree}{\fUD{Open Flags}{SEGMEXEC}}{dark}
        \begin{posnex}
# Enable SEGMEXEC for an executable
# This allows execution of code in the data segment
paxctl -S busybox

# Disable SEGMEXEC for an executable
# This disables execution of code in the data segment
paxctl -s busybox
        \end{posnex}
    \end{baseBoxThree}
    \smallskip
\end{baseBoxThree}

\begin{baseBoxThree}{\fU{PaX}}{dark}
    \smallskip
    \begin{baseBoxThree}{Wrap Up}{dark}
        By integrating PaX, Stack Smashing Protection, and Position Independent Executables, Alpine Linux ensures a robust security posture against various types of attacks.
        These features help to mitigate the risks of memory corruption, buffer overflows, and code injection attacks, providing a secure environment for running applications.
    \end{baseBoxThree}
    \smallskip
\end{baseBoxThree}
