\phantomsection{}
\label{Control Flow Integrity (CFI)}
\addcontentsline{toc}{subsection}{Control Flow Integrity (CFI)}
\begin{baseBoxThree}{\sL{Control Flow Integrity (CFI)}}{dark}
    \bigskip
    \sTE{Description}: Control Flow Integrity (CFI) is a security technique designed to ensure that the control flow of a program remains unaltered during execution.
    It prevents attackers from hijacking the program's control flow, which is a common goal in many exploits, such as buffer overflow attacks.
    By enforcing CFI, the program can only execute legitimate control flow paths, making it much harder for attackers to execute arbitrary code.
    \bigskip
    \begin{baseBoxThree}{How CFI Works}{dark}
        \smallskip
        \begin{posnexItemize}
            \item[\sA] \sE{Instrumenting Code}: Adding checks to the program's code to ensure that indirect jumps and calls only go to valid targets.
            \item[\sA] \sE{Creating Control Flow Graphs (CFG)}: Mapping out all possible paths the program can take during execution.
            \item[\sA] \sE{Checking Transfers}: At runtime, verifying that control transfers (like function calls and returns) follow the paths defined in the CFG.
        \end{posnexItemize}
        \smallskip
    \end{baseBoxThree}
    \smallskip
\end{baseBoxThree}

\begin{baseBoxThree}{\fU{Control Flow Integrity (CFI)}}{dark}
    \smallskip
    \phantomsection{}
    \label{CFI Example Script}
    \begin{baseBoxThree}{CFI Example Script}{dark}
        \smallskip
        \begin{posnex}
apk update && apk upgrade
apk add clang build-base
cd /tmp

{
cat << 'EOF'
#include <stdio.h>
#include <stdlib.h>
#include <stdint.h>

void target_function() {
        printf("Target function executed.\n");
}

void vulnerable_function(void (*func)()) {
        func();
}

int main(int argc, char **argv) {
        if (argc != 2) {
                fprintf(stderr, "Usage: %s <input>\n", argv[0]);
                return 1;
        }

        void (*func)() = NULL;

        if (argv[1][0] == '1')
                func = target_function;
        else
                func = (void (*)())(uintptr_t)strtoul(argv[1], NULL, 0);

        vulnerable_function(func);

        return 0;
}
EOF
} > example.c

TMPA="$(clang -fsanitize=cfi \
    -flto -fvisibility=hidden \
    -o cfi_demo example.c 2>&1 | awk -F "'" '{print $2}')"

mkdir -p $(dirname "${TMPA}")
# Continues on the next page...
        \end{posnex}
        \smallskip
    \end{baseBoxThree}    
    \smallskip
\end{baseBoxThree}

\begin{baseBoxThree}{\fU{Control Flow Integrity (CFI)}}{dark}
    \begin{baseBoxThree}{\fH{CFI Example Script}{\faAngleUp~CFI Example Script}}{dark}
        \begin{posnex}
{
cat << 'EOF'
fun:func
EOF
} > "${TMPA}"

unset TMPA

clang -fsanitize=cfi -flto \
    -fvisibility=hidden \
    -o cfi_demo example.c

# Valid input (target_function will be executed)
./cfi_demo 1 # Output: Target function executed.

# Invalid input (CFI should detect control flow hijacking)
./cfi_demo 0x12345678

objdump -d cfi_demo | grep '.cfi'
        \end{posnex}
    \end{baseBoxThree}
    \smallskip
    \begin{baseBoxThree}{Note}{info}
        \smallskip
        While GCC has some support for CFI, Clang's implementation of CFI is more mature and reliable.
        Clang provides better instrumentation and runtime checks, making it a preferred choice for enabling advanced security features like CFI.
        This is why we used Clang in our demonstration to ensure robust and effective protection against control flow hijacking attacks.
        \smallskip
    \end{baseBoxThree}
    \smallskip
    \begin{baseBoxThree}{Wrap Up}{dark}
        \smallskip
        We also demonstrated how to compile a C program with CFI enabled using Clang, and verified the presence of CFI instrumentation in the binary using objdump.
        \smallskip
    \end{baseBoxThree}
    \smallskip
\end{baseBoxThree}
