\phantomsection{}
\label{DAC vs MAC}
\addcontentsline{toc}{subsection}{DAC vs MAC}
\begin{baseBoxThree}{DAC vs MAC}{dark}
    \bigskip
    \begin{baseBoxThree}{}{dark}
        \begin{posnexItemize}
            \item[\sA] \sE{DAC}: User-based control, flexible, commonly used in everyday applications.
            \item[\sA] \sE{MAC}: Central authority control, strict policies, used in high-security environments.
        \end{posnexItemize}
    \end{baseBoxThree}
    \bigskip
    \phantomsection{}
    \label{DAC (Discretionary Access Control)}
    \addcontentsline{toc}{subsubsection}{DAC (Discretionary Access Control)}
    \begin{baseBoxThree}{DAC (Discretionary Access Control)}{dark}
        \smallskip
        \begin{baseBoxThree}{}{dark}
            \smallskip
            Discretionary Access Control (DAC) is a type of access control where the data owner decides who can access specific resources. It's called "discretionary" because it's at the owner's discretion. Here's what you need to know:
            \smallskip
        \end{baseBoxThree}
        \smallskip
        \begin{posnexItemize}
            \item[\sA] \sE{User Control}: The owner of the resource (e.g., a file or folder) has the ability to determine who can access it.
            \item[\sA] \sE{Flexible}: Provides flexibility in setting permissions. Users can grant and revoke access to others.
            \item[\sA] \sE{Common in Applications}: Frequently used in applications where user-defined permissions are common, like in file systems.
        \end{posnexItemize}
    \end{baseBoxThree}
    \smallskip
    \phantomsection{}
    \label{MAC (Mandatory Access Control)}
    \addcontentsline{toc}{subsubsection}{MAC (Mandatory Access Control)}    
    \begin{baseBoxThree}{MAC (Mandatory Access Control)}{dark}
        \smallskip
        \begin{baseBoxThree}{}{dark}
            \smallskip
            Mandatory Access Control (MAC) is more restrictive and is often used in environments where security is critical. Access decisions are made by a central authority based on predefined security policies. Key features include:
            \smallskip
        \end{baseBoxThree}
        \smallskip
        \begin{posnexItemize}
            \item[\sA] \sE{Centralized Control}: Access rights are governed by a central authority, not by the individual user.
            \item[\sA] \sE{Security Labels}: Resources and users are assigned security labels (e.g., "confidential," "top secret"). Access decisions are made based on these labels.
            \item[\sA] \sE{Strict Policies}: Typically used in environments that require a high level of security, such as military or government organizations.
        \end{posnexItemize}
        \smallskip
    \end{baseBoxThree}
    \smallskip
\end{baseBoxThree}
